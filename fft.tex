\documentclass{article}
\usepackage[utf8x]{inputenc} 
\usepackage[croatian]{babel}

\usepackage{amssymb}
\usepackage{amsmath}

\usepackage{listings}
\usepackage{mathtools}

\title{Fast Fourier transform}
\begin{document}

\section{Fourierov red}
Niz funkcija: 
\begin{align*}
1,\quad cos \left( \frac{2 k \pi x}{l} \right),\quad sin \left(\frac{2 k \pi x}{l}\right)\quad  \text{$k \in \mathbb{N}$}
\end{align*}
čini bazu prostora $H\left[ \frac{−l}{2}, \frac{l}{2}\right]$ sa standardnim skalarnim produktom. \\
Njoj pridužen rastav funkcije $f \in H\left[ \frac{−l}{2}, \frac{l}{2}\right]$ nazivamo
trigonometrijskim Fourierovim redom:
\begin{align} \label{red}
 f(x) = \frac{a_0}{2} + \sum_{k=1}^{\infty} \left( a_k cos \frac{2 k \pi x}{l} + b_k sin \frac{2k \pi x}{l} \right) 
\end{align}
uz koeficijente:
\[\begin{aligned}
a_{k} &=\frac{1}{l} \int_{\frac{-l}{2}}^{\frac{l}{2}} f(x) \cos \frac{2 k \pi x}{l}, \quad k=0,1,2 \ldots \\
b_{k} &=\frac{1}{l} \int_{\frac{-l}{2}}^{\frac{l}{2}} f(x) \sin \frac{2 k \pi x}{l}, \quad k=1,2 \ldots
\end{aligned}\]

\section{Fourierova transformacija}
Cilj je proširiti rezultat prvog dijela na funkcije definirane na cijelom $\mathbb{R}$.\\ 
Koristeći Eulerov identitet:

\[ r e^{i\phi}=r(\cos{\phi}+i\sin{\phi}).\]
dobivamo izraz ekivalentan (\ref{red}):
\begin{align} \label{transformacija}
 f(x)=\sum_{k=-\infty}^{\infty}{c_{k} e^{ \frac{2k\pi i}{l}x} }, 
\end{align}
\[c_{k}=\frac{1}{l}\int_{\frac{-l}{2}}^{\frac{l}{2}}{f(x)e^{-\frac{2k\pi i}{l}x}dx}\] \\
Puštanjem $l \rightarrow \infty$ koeficijenti $c_k$ teže u $0$.\\
Međutim, promotrimo li $c_k$ kao funkcije ovisne o $\frac{k}{l}$ dobivamo:
\[
F\left(\frac{k}{l}\right)=\int_{\frac{-l}{2}}^{\frac{l}{2}}{f(x)e^{-\frac{2k\pi i}{l}x}dx}
\]
Tada nova formula za Fourierov red glasi:
\[
f(x)=\sum_{k=-\infty}^{\infty}{\frac{1}{l}F\left(\frac{k}{l}\right) e^{\frac{k}{l} 2\pi i x }}.
\]
Puštanjem $l \rightarrow \infty$ dobivamo izraz za Fourierovu transformaciju:
\[
f(x)= \lim_{l \to \infty} \sum_{k=-\infty}^{\infty}{\frac{1}{l}F\left(\frac{k}{l}\right) e^{\frac{k}{l} 2\pi ix}} = \int_{- \infty}^{ \infty}{F(s)e^{s 2\pi ix}ds}
\]

\[
F(s)=\int_{-\infty}^{\infty}{f(x)e^{-s 2\pi ix}dx}
\]


\section{Diskretna Fourierova transformacija (DFT)}
U stvarnim primjenama često nemamo zatvorenu formulu za funkciju $f$, nego ju aproksimiramo uzorkovanjem na diskretnom skupu $x_{0},\ldots,x_{n-1}$.\\
U tom nam slučaju Fourierova transformacija (koja je definirana za kontinuiran skup) ne odgovara pa za $f(x_{k})=f_{k}$ definiramo DFT na sljedeći način:
\begin{align} \label{dft}
F_{j} = \sum_{k=0}^{n-1} f_{k} e^{\frac{2\pi ikj}{n}}, \quad j=0,1,2 \ldots n-1    
\end{align}
Analogno za inverznu transformaciju i funkciju $F$ uzorkovanu na skupu $s_{0},\ldots,s_{n-1}$ imamo:
\begin{align}
f_{k} =\frac{1}{n}\sum_{j=0}^{n-1}F_{j} e^{\frac{-2\pi ikj}{n}}, \quad k=0,1,2 \ldots n-1    
\end{align}
\end{document}


